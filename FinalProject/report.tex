\documentclass[a4paper, 12pt]{article}
\usepackage[utf8]{inputenc}
\usepackage{url}
\usepackage{amsmath, amssymb, amsfonts}
\usepackage{graphicx}

\begin{document}
\begin{center}
{\LARGE \bf Praktikum Wissenschaftliches Rechnen \\ \vspace{0.2cm} (CFD, Final Project)}\\
\vspace{0.6cm}
{\small Group 9: Breznik, E., Cheng, Z., Ni, W., Schmidbartl, N.}
\end{center}
\normalfont
\section{The Topic}

\section{Implementation}
\subsection{Boundary conditions for the velocities}
We have 5 different boundary conditions, namely no-slip, free-slip, outflow, inflow, moving wall. We implement them for the 6 boundaries of our 3D domain, as well as for internal boundaries.
\subsubsection{No-Slip boundary condition}
For no-slip conditions, the fluid vanishes at the boundary. As a consequence, both the velocity component normal to and parallel to the boundary are zero. Using staggered grid, the discrete velocity components normal to and parallel to the boundary lie directly at the boundary, whilst the one parallel to the boundary is the average of the boundary cell's and its neighbouring fluid cell's velocity components.

Therefore, for the \texttt{B\_O} case, we set the following boundary conditions: 
\begin{equation}
\begin{array}{lll}
u_{i,j,k} = 0, & v_{i,j-1,k} = -v_{i+1,j-1,k}, & v_{i,j,k} = -v_{i+1,j,k}, \\
w_{i,j,k-1} = -w_{i+1,j,k-1}, & w_{i,j,k} = -w_{i+1,j,k},& \\
\end{array}
\end{equation}
and we analogously set the boundary conditions for the following 5 cases: \texttt{B\_W}, \texttt{B\_N}, \texttt{B\_S}, \texttt{B\_U} and \texttt{B\_D}.

For the \texttt{B\_NO} case, we set the following boundary conditions:
\begin{equation}
\begin{array}{ll}
u_{i,j,k} = 0, & u_{i-1,j,k} = -u_{i-1,j+1,k}, \\
v_{i,j,k} = 0, & v_{i,j-1,k} = -v_{i+1,j-1,k}, \\
w_{i,j,k} = -\frac{1}{2}(w_{i+1,j,k}+w_{i,j+1,k}), &
w_{i,j,k-1} = -\frac{1}{2}(w_{i+1,j,k-1}+w_{i,j+1,k-1}), \\
\end{array}
\end{equation}
and we analogously set the boundary conditions for the following 11 cases: \texttt{B\_NW}, \texttt{B\_NU}, \texttt{B\_ND}, \texttt{B\_SO}, \texttt{B\_SW}, \texttt{B\_SU}, \texttt{B\_SD}, \texttt{B\_OU}, \texttt{B\_WU}, \texttt{B\_OD} and \texttt{B\_WD}.

For the \texttt{B\_NOU} case, we set the following boundary conditions:
\begin{equation}
\begin{array}{ll}
u_{i,j,k} = 0, & u_{i-1,j,k} = -\frac{1}{2}(u_{i-1,j+1,k}+u_{i-1,j,k+1}), \\
v_{i,j,k} = 0, & v_{i,j-1,k} = -\frac{1}{2}(v_{i+1,j-1,k}+v_{i+1,j-1,k+1}), \\
w_{i,j,k} = 0, & w_{i,j,k-1} = -\frac{1}{2}(w_{i+1,j,k-1}+w_{i,j+1,k-1}), \\
\end{array}
\end{equation}
and we analogously set the boundary conditions for the following 7 cases: \texttt{B\_NWU}, \texttt{B\_NOD}, \texttt{B\_NWD}, \texttt{B\_SOU}, \texttt{B\_SWU}, \texttt{B\_SOD}, and \texttt{B\_SWD}.

\subsubsection{Free-Slip boundary condition}
For free-slip conditions, the fluid flows freely parallel to the boundary, but does not cross the boundary. As a consequence, the velocity component normal to the boundary is zero, as well as the normal derivative of the velocity component parallel to the wall. Using staggered grid, the discrete velocity components normal to the boundary lie directly at the boundary, whilst the one parallel to the boundary is the average of the boundary cell's and its neighbouring fluid cell's velocity components.

Therefore, for the \texttt{B\_O} case, we set the following boundary conditions:
\begin{equation}
\begin{array}{lll}
u_{i,j,k} = 0, & v_{i,j-1,k} = v_{i+1,j-1,k}, & v_{i,j,k} = v_{i+1,j,k}, \\
w_{i,j,k-1} = w_{i+1,j,k-1}, & w_{i,j,k} = w_{i+1,j,k},& \\
\end{array}
\end{equation}
and we analogously set the boundary conditions for the following 5 cases: \texttt{B\_W}, \texttt{B\_N}, \texttt{B\_S}, \texttt{B\_U} and \texttt{B\_D}.

For the \texttt{B\_NO} case, we set the following boundary conditions:
\begin{equation}
\begin{array}{ll}
u_{i,j,k} = 0, & u_{i-1,j,k} = u_{i-1,j+1,k}, \\
v_{i,j,k} = 0, & v_{i,j-1,k} = v_{i+1,j-1,k}, \\
w_{i,j,k} = \frac{1}{2}(w_{i+1,j,k}+w_{i,j+1,k}), &
w_{i,j,k-1} = \frac{1}{2}(w_{i+1,j,k-1}+w_{i,j+1,k-1}), \\
\end{array}
\end{equation}
and we analogously set the boundary conditions for the following 11 cases: \texttt{B\_NW}, \texttt{B\_NU}, \texttt{B\_ND}, \texttt{B\_SO}, \texttt{B\_SW}, \texttt{B\_SU}, \texttt{B\_SD}, \texttt{B\_OU}, \texttt{B\_WU}, \texttt{B\_OD} and \texttt{B\_WD}.

For the \texttt{B\_NOU} case, we set the following boundary conditions:
\begin{equation}
\begin{array}{ll}
u_{i,j,k} = 0, & u_{i-1,j,k} = \frac{1}{2}(u_{i-1,j+1,k}+u_{i-1,j,k+1}), \\
v_{i,j,k} = 0, & v_{i,j-1,k} = \frac{1}{2}(v_{i+1,j-1,k}+v_{i+1,j-1,k+1}), \\
w_{i,j,k} = 0, & w_{i,j,k-1} = \frac{1}{2}(w_{i+1,j,k-1}+w_{i,j+1,k-1}), \\
\end{array}
\end{equation}
and we analogously set the boundary conditions for the following 7 cases: \texttt{B\_NWU}, \texttt{B\_NOD}, \texttt{B\_NWD}, \texttt{B\_SOU}, \texttt{B\_SWU}, \texttt{B\_SOD}, and \texttt{B\_SWD}.

\subsubsection{Outflow boundary condition}
For outflow conditions, the normal derivatives of both the velocity components are zero. Using staggered grid, the discrete velocity component normal to the boundary lie directly at the boundary, whilst the one parallel to the boundary is the average of the boundary cell's and its neighbouring fluid cell's velocity components.

Therefore, for the \texttt{B\_O} case, we set the following boundary conditions:
\begin{equation}
\begin{array}{lll}
u_{i,j,k} = u_{i+1,j,k}, & v_{i,j-1,k} = v_{i+1,j-1,k}, & v_{i,j,k} = v_{i+1,j,k}, \\
w_{i,j,k-1} = w_{i+1,j,k-1}, & w_{i,j,k} = w_{i+1,j,k},& \\
\end{array}
\end{equation}
and we analogously set the boundary conditions for the following 5 cases: \texttt{B\_W}, \texttt{B\_N}, \texttt{B\_S}, \texttt{B\_U} and \texttt{B\_D}.

For the \texttt{B\_NO} case, we set the following boundary conditions:
\begin{equation}
\begin{array}{ll}
u_{i,j,k} = u_{i+1,j,k}, & u_{i-1,j,k} = u_{i-1,j+1,k}, \\
v_{i,j,k} = v_{i,j+1,k}, & v_{i,j-1,k} = v_{i+1,j-1,k}, \\
w_{i,j,k} = \frac{1}{2}(w_{i+1,j,k}+w_{i,j+1,k}), &
w_{i,j,k-1} = \frac{1}{2}(w_{i+1,j,k-1}+w_{i,j+1,k-1}), \\
\end{array}
\end{equation}
and we analogously set the boundary conditions for the following 11 cases: \texttt{B\_NW}, \texttt{B\_NU}, \texttt{B\_ND}, \texttt{B\_SO}, \texttt{B\_SW}, \texttt{B\_SU}, \texttt{B\_SD}, \texttt{B\_OU}, \texttt{B\_WU}, \texttt{B\_OD} and \texttt{B\_WD}.

For the \texttt{B\_NOU} case, we set the following boundary conditions:
\begin{equation}
\begin{array}{ll}
u_{i,j,k} = u_{i+1,j,k}, & u_{i-1,j,k} = \frac{1}{2}(u_{i-1,j+1,k}+u_{i-1,j,k+1}), \\
v_{i,j,k} = v_{i,j+1,k}, & v_{i,j-1,k} = \frac{1}{2}(v_{i+1,j-1,k}+v_{i+1,j-1,k+1}), \\
w_{i,j,k} = w_{i,j,k+1}, & w_{i,j,k-1} = \frac{1}{2}(w_{i+1,j,k-1}+w_{i,j+1,k-1}), \\
\end{array}
\end{equation}
and we analogously set the boundary conditions for the following 7 cases: \texttt{B\_NWU}, \texttt{B\_NOD}, \texttt{B\_NWD}, \texttt{B\_SOU}, \texttt{B\_SWU}, \texttt{B\_SOD}, and \texttt{B\_SWD}.

\subsubsection{Inflow boundary condition}
For inflow conditions, we assume the inflow velocity is perpendicular to the inflow boundary. We implement the following 6 cases, \texttt{B\_O}, \texttt{B\_W}, \texttt{B\_N}, \texttt{B\_S}, \texttt{B\_U} and \texttt{B\_D}, and forbid the other cases.

For the \texttt{B\_O} case, the boundary velocities are same with those the no-slip condition, except that $u_{i,j,k} = \texttt{velIN}$, where the inflow velocity \texttt{velIN} is larger than zero.

Similarly, for the \texttt{B\_W} case, the boundary velocities are same with those of the no-slip condition, except that $u_{i,j,k} = -\texttt{velIN}$, where \texttt{velIN} is larger than zero, and we analogously set the boundary conditions for the following 4 cases: \texttt{B\_N}, \texttt{B\_S}, \texttt{B\_U} and \texttt{B\_D}.

\subsubsection{Moving wall boundary condition}
For moving wall conditions, again we need to restrict the moving wall directions.

When the boundary cell is like \texttt{B\_O}, we set the moving wall direction to the direction of next (first nonfixed) coordinate, i.e., if $x/y/z$ is fixed, the wall moving in $y/z/x$. For example, if the flag is \texttt{B\_O}, the moving wall direction is $+y$ or $-y$. Consequently, the velocities are same with those of the no-slip condition, except for the ones along this moving wall direction and parallel to the boundary. Therefore, for the \texttt{B\_O} case, the following boundary conditions differ from those of the no-slip condition:
\begin{equation}
\begin{array}{ll}
v_{i,j-1,k} = 2*\texttt{velMW}_y-v_{i+1,j-1,k}, & v_{i,j,k} = 2*\texttt{velMW}_y-v_{i+1,j,k}
\end{array}
\end{equation}
, where $\texttt{velMW} = [\texttt{velMW}_x, \texttt{velMW}_y, \texttt{velMW}_z]$ is the moving wall velocity vector, and $\texttt{velMW}_x$ is its component along the $x$ axis. We analogously set the boundary conditions for the following 5 cases: \texttt{B\_W}, \texttt{B\_N}, \texttt{B\_S}, \texttt{B\_U} and \texttt{B\_D}.

When the boundary cell is like \texttt{B\_NO}, we set the directions of moving walls \texttt{N} and \texttt{O} in a ``circular'' way. For example, if \texttt{N} moves in the $+x$ direction, then \texttt{O} moves in the $-y$ direction. Alternatively, if \texttt{N} moves in the $-x$ direction, then \texttt{O} moves in the $+y$ direction. Therefore, for the \texttt{B\_NO} case, the following boundary conditions differ from those of the no-slip condition:
\begin{equation}
\begin{array}{ll}
u_{i,j,k} = 2*\texttt{velMW}_x-u_{i,j+1,k}, & u_{i-1,j,k} = 2*\texttt{velMW}_x-u_{i-1,j+1,k}, \\
v_{i,j,k} = 2*\texttt{velMW}_y-v_{i+1,j,k}, & v_{i,j-1,k} = 2*\texttt{velMW}_y-v_{i+1,j-1,k}
\end{array}
\end{equation}
and we analogously set the boundary conditions for the following 11 cases: \texttt{B\_NW}, \texttt{B\_NU}, \texttt{B\_ND}, \texttt{B\_SO}, \texttt{B\_SW}, \texttt{B\_SU}, \texttt{B\_SD}, \texttt{B\_OU}, \texttt{B\_WU}, \texttt{B\_OD} and \texttt{B\_WD}.

However, for the moving wall condition we forbid the boundary conditions for the following 8 cases: \texttt{B\_NOU}, \texttt{B\_NWU}, \texttt{B\_NOD}, \texttt{B\_NWD}, \texttt{B\_SOU}, \texttt{B\_SWU}, \texttt{B\_SOD}, and \texttt{B\_SWD}.

\subsection{Boundary condition for the pressure}
On the other hand, for all the 5 aforementioned boundary conditions, the boundary values for the pressure are derived from the discretized momentum equation and result in discrete \textit{Neumann} conditions.

Therefore, for the \texttt{B\_O} case, we set the following boundary conditions:
\begin{equation}
\begin{array}{lll}
F_{i,j,k} = u_{i,j,k}, & p_{i,j,k} = p_{i+1,j,k} & 
\end{array}
\end{equation}
and we analogously set the boundary conditions for the following 5 cases: \texttt{B\_W}, \texttt{B\_N}, \texttt{B\_S}, \texttt{B\_U} and \texttt{B\_D}.

For the \texttt{B\_NO} case, we set the following boundary conditions
\begin{equation}
\begin{array}{ll}
F_{i,j,k} = u_{i,j,k}, \quad G_{i,j,k} = v_{i,j,k},
& p_{i,j,k} = \frac{1}{2}(p_{i+1,j,k}+p_{i,j+1,k}))
\end{array}
\end{equation}
and we analogously set the boundary conditions for the following 11 cases: \texttt{B\_NW}, \texttt{B\_NU}, \texttt{B\_ND}, \texttt{B\_SO}, \texttt{B\_SW}, \texttt{B\_SU}, \texttt{B\_SD}, \texttt{B\_OU}, \texttt{B\_WU}, \texttt{B\_OD} and \texttt{B\_WD}.

For the \texttt{B\_NOU} case, we set the following boundary conditions
\begin{equation}
\begin{array}{ll}
F_{i,j,k} = u_{i,j,k}, & G_{i,j,k} = v_{i,j,k}, \quad H_{i,j,k} = w_{i,j,k}, \\
 & p_{i,j,k} = \frac{1}{3}(p_{i+1,j,k}+p_{i,j+1,k}+p_{i,j,k+1})
\end{array}
\end{equation}
and we analogously set the boundary conditions for the following 7 cases: \texttt{B\_NWU}, \texttt{B\_NOD}, \texttt{B\_NWD}, \texttt{B\_SOU}, \texttt{B\_SWU}, \texttt{B\_SOD}, and \texttt{B\_SWD}.

\section{Problems, current state and future work}

\begin{thebibliography}{99}
\bibitem{Griebel}
Griebel, M., Dornsheifer, T., Neunhoeffer, T.: \emph{Numerical Simulation in 
Fluid Dynamics: A Practical Introduction}. SIAM, {\bf 1998}.

\bibitem{Hirt}
Hirt, C. W., Nichols, B. D.: \emph{Volume of Fluid Method for the Dynamics of Free Boundaries}. Journal of Computational Physics {\bf 39} (1981).

\bibitem{sola}
Hirt, C. W., Nichols, B. D., Hotchkiss, R. S.: \emph{SOLA-VOF: A solution Algorithm for Transient Fluid Flow with Multiple Free Boudaries}. LASL, {\bf 1980}. 
\end{thebibliography}

\end{document}
